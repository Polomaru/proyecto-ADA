\documentclass{article}
\usepackage[spanish]{babel}
\usepackage[utf8]{inputenc}
\usepackage{mathtools}
\usepackage{graphicx}
\usepackage{algorithm}
\usepackage[noend]{algpseudocode}
\usepackage{bm}


\begin{document}
\begin{titlepage}
    \begin{center}
        \vspace*{1cm}
            
        \Huge
        Proyecto - Entrega 1
            
        \vspace{1cm}
        \Large
        Domínguez Aspilcueta, Pedro Francisco - 201910375\\
            
        \vfill
            
        \includegraphics[width=0.4\textwidth]{logo.jpg}
            
        \normalsize
        \textbf{Universidad de Ingeniería y Tecnología}\\
        Ciencia de la computación\\
        Análisis y Diseño de Algoritmos 1.00\\
        Docente: Gutierrez Alva, Juan Gabriel\\
        TA: Lopez Condori, Rodrigo\\
    \end{center}
\end{titlepage}
    
\section*{Pregunta 2}

\vspace{0.5cm}

\noindent RECIBE: $a_{1}$ de una cadena no vacía $s=\{a_1a_2\cdots a_n\}$ \\
DEVUELVE: Una cadena $r=\{f_k(a_1)f_k(a_2)\cdots f_k(a_n)\}$ tal que $1 \leq k \leq 4$ y r es
de tamaño mínimo óptimo
\begin{algorithmic}[1]
\item[]{OPT($i$, $j$)}
\If{($a_i \in C_j)$}
\State $r=r+f_j(a_i)$
\Else 
\State $r=r+``\cdots"$
\EndIf
\If{$i = i_{ultimo}$}
\State \textbf{return} r
\EndIf
\State $t_j = ``"$
\State $t_{(j+1)\%4} = T_{(j+1)\%4}$
\State $t_{(j+2)\%4} = T_{(j+2)\%4}$
\State $t_{(j+3)\%4} = T_{(j+3)\%4}$
\State $i = i+1$
\State $r=r+min(r+t_1+OPT(i,1),r+t_2+OPT(i,2),
r+t_3+OPT(i,3), r+t_4+OPT(i,4))$
\State \textbf{return r} 
\end{algorithmic}

\vspace{1cm}

En la primera recurrencia se calcula la codificación del caracter $s_0$ para posteriormente concatenarlo
con la mínima longitud de las 4 posibles codificaciones $C_1,C_2,C_3,C_4$ para $s_1$.\\
Esta recurrencia se realiza para toda la cadena $s$ hasta llegar al caso base $s_n$ donde se retorna $f_n(a_n)$
siendo así que se calcula todas las posibles combinaciones y se garantiza que se obtiene la solución óptima.\\

\textbf{Complejidad}\\
Si existen $C_1,C_2,C_3,C_4$ y se calculan todas las posibles codificaciones para todo caracter de $s$.
Por lo tanto, el algoritmo tiene una complejidad de $O(4^n)$ donde $n$ es la longitud de la cadena.


\end{document}