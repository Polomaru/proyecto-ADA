\documentclass{article}
\usepackage[spanish]{babel}
\usepackage[utf8]{inputenc}
\usepackage{mathtools}
\usepackage{graphicx}
\usepackage{amsmath}
\usepackage{algorithm}
\usepackage[noend]{algpseudocode}
\usepackage{bm}


\begin{document}

\begin{titlepage}
    \begin{center}
        \vspace*{1cm}
            
        \Huge
        Proyecto - Entrega 1
            
        \vspace{1cm}
        \Large
        Domínguez Aspilcueta, Pedro Francisco - 201910375\\
            
        \vfill
            
        \includegraphics[width=0.4\textwidth]{logo.jpg}
            
        \normalsize
        \textbf{Universidad de Ingeniería y Tecnología}\\
        Ciencia de la computación\\
        Análisis y Diseño de Algoritmos 1.00\\
        Docente: Gutierrez Alva, Juan Gabriel\\
        TA: Lopez Condori, Rodrigo\\
    \end{center}
\end{titlepage}    

\begin{itemize}
    \item Sea $s$ una cadena no vacía tal que $s = \{a_1a_2\cdots a_n\}$. 
    \item $\{C_0, C_1, C_2, C_3\}$ los modos de codificar $a$.
    \item $B$ una cadena de dos bits $\{00, 01, 10, 11\}$ que indica cual $C$ comienza una 
    codificación de $s$.
    \item $\{T_0, T_1, T_2, T_3\}$ las cadenas de transición a otro $C$. Tomando en cuenta que 
    $T$ puede ser vacío si no se requiere ninguna transición.
    \item $r$ una codificación de $s$ tal que $r=\{B \cdot C(a_1) \cdot T \cdot C(a_2) \cdot T 
    \cdot C(a_3) T \cdots T \cdot C(a_n)\}$.
\end{itemize}

\section{Pregunta 2}
\[ OPT(i, j) \begin{cases} 
    B_j + C_j(a_i) & i = 1; a_i \subset C_j \\
    \infty & a_i \not\subset C_j \\
    min\{OPT(i - 1, j) + C_j(a_i), & i > 1\\
    OPT(i - 1, (j+1) \%4) + T_{(j+1) \%4} + C_j(a_i), \\
    OPT(i - 1, (j+2) \%4) + T_{(j+2) \%4} +C_j(a_i), \\ OPT(i - 1, (j+3) \%4) + T_{(j+3) \%4} + C_j(a_i)\}
 \end{cases}
\]

\section{Pregunta 3}

\noindent RECIBE: un índice final $i$ de una cadena $s$ y un índice $j$ de $C$.\\
DEVUELVE: Una cadena $r$ que codifica a $s$ de manera mínima óptima con $C_j(a_i)$.

\begin{algorithmic}[1]
\item[]{OPT($i$, $j$)}
\If{($a_i \subset C_j)$ and $i = 1$}
\State \textbf{return} $B_j + C_j(a_i)$
\ElsIf {($a_i \not\subset C_j)$}
\State \textbf{return} $\infty$
\EndIf
\State $A_0 = C_j(a_i)$
\State $A_1 = T_{(j+1)\%4} + C_j(a_i)$
\State $A_2 = T_{(j+2)\%4} + C_j(a_i)$
\State $A_3 = T_{(j+3)\%4} + C_j(a_i)$
\State $r=min(OPT(i-1,j) + A_0, OPT(i-1,(j+1)\%4) + A_1, OPT(i-1,(j+2)\%4) + A_2, OPT(i-1,(j+3)\%4) + A_3)$
\State \textbf{return r} 
\end{algorithmic}

\newpage

\noindent RECIBE: un cadena $s$ no vacía.\\
DEVUELVE: Una cadena que codifica a $s$ de manera mínima óptima.
\begin{algorithmic}[1]
\item[]{MIN-COD($s$)}
\State Sea $a_n$ el caracter final de $s$.  
\State \textbf{return} $min(OPT(a_n,0), OPT(a_n,1), OPT(a_n,2), OPT(a_n,3))$  
\end{algorithmic}

En la primera recurrencia se calcula la codificación del caracter $s_0$ para posteriormente concatenarlo
con la mínima longitud de las 4 posibles codificaciones $C_1,C_2,C_3,C_4$ para $s_1$.\\
Esta recurrencia se realiza para toda la cadena $s$ hasta llegar al caso base $s_n$ donde se retorna $f_n(a_n)$
siendo así que se calcula todas las posibles combinaciones y se garantiza que se obtiene la solución óptima.\\

\textbf{Complejidad}\\
Si existen $C_1,C_2,C_3,C_4$ y se calculan todas las posibles codificaciones para todo caracter de $s$.
Por lo tanto, el algoritmo tiene una complejidad de $O(4^n)$ donde $n$ es la longitud de la cadena.



\subsection{Pregunta 5}
\noindent RECIBE: una cadena $s$.\\
DEVUELVE: Una cadena $r$ que codifica a $s$ de manera mínima óptima.

\begin{algorithmic}[1]
\item[]{MIN-COD-DIN($s$)}
\State Sea $A$ una matriz que guarda la solución.
\State $B=\{00, 01, 10, 11\}$ tal que que indica cual es el modo de codificación inicial.
\For {j=1 to 4}
    \State $A[0][j] = B_j$
\EndFor
\For{i = 1 to n}
    \State $A[i][1] = \infty$
    \State $A[i][2] = \infty$
    \State $A[i][3] = \infty$
    \State $A[i][4] = \infty$
    \For{j = 1 to 4}
        \If{Si $a_i$ puede ser codificado con $C_j$}
            \State $M_1 = A[i-1][j] + C_j(a_i)$
            \State $M_2 = A[i-1][(j+1)\%4] + T_{(j+1)\%4} + C_{j}(a_i)$
            \State $M_3 = A[i-1][(j+2)\%4] + T_{(j+2)\%4} + C_{j}(a_i)$
            \State $M_4 = A[i-1][(j+3)\%4] + T_{(j+3)\%4} + C_{j}(a_i)$
            \State $A[i][j] = min(M_1, M_2, M_3, M_4)$
            \State
        \EndIf
    \EndFor
\EndFor
\State $r = min(A[n][1], A[n][2], A[n][3], A[n][4])$
\State \textbf{return r}  
\end{algorithmic}

\end{document}