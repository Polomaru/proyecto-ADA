\section{Problema de codificación de texto}

\paragraph{Problema MIN-COD}
Dada una cadena $s$ sobre el conjunto $C$, encontrar una codificación de $C$ óptima.

\begin{itemize}
    \item Sea $s$ una cadena no vacía tal que $s = \{a_1a_2\cdots a_n\}$. 
    \item $\{C_0, C_1, C_2, C_3\}$ los modos de codificar $a$.
    \item $B$ una cadena de dos bits $\{00, 01, 10, 11\}$ que indica cual $C$ comienza una 
    codificación de $s$.
    \item $\{T_0, T_1, T_2, T_3\}$ las cadenas de transición a otro $C$. Tomando en cuenta que 
    $T$ puede ser vacío si no se requiere ninguna transición.
\end{itemize}

Podemos suponer que una cadena $s$ siempre puede ser codificada por al menos un modo de 
codificación $C$. Y decimos que $r$ es una  solución óptima tal que $r$ empieza con una cadena 
$B$ de dos bits seguido de un número de  codificaciones y transiciones mínimos. 
$r=\{B \cdot C(a_1) \cdot T \cdot C(a_2) \cdot T \cdot C(a_3) T \cdots T \cdot C(a_n)\}$

\subsection{Pregunta 1}

\paragraph{Heurística Voraz:}
Utilizar el modo de codificación que permita una codificación mínima para los caracteres 
$a_i$ y $a_{i+1}$.\\

\noindent RECIBE: un cadena $s$ no vacía.\\
DEVUELVE: Una cadena $r$ de bits que codifica a $s$ y su tamaño.

\begin{algorithmic}[1]
\item[]{VORAZ($s$)}
\For {i = 0 to n - 1}
\EndFor
\If{n es par}
\EndIf
\end{algorithmic}

\subsection{Pregunta 2}

Sea $X$ una solución óptima para el problema. Existen tres casos dependiendo de la codificación 
del último caracter $a_n$ de $s$.

\begin{itemize}
    \item \textbf{Caso 1:} $C_j(a_n) \not\in X$. 
    En ese caso, $a_n$ no puede ser codificado por el modo $C_j$. Por lo que no existe una 
    codificación óptima que incluya a $C_j(a_n)$.  
    \item \textbf{Caso 2:} $C_j(a_n) \in X$.
    En ese caso, $a_n$ puede ser codificado por el modo $C_j$. Considere que $X' = X / \{a_n\}$.
    Note que $X'$ es una solución óptima para el subproblema $s={a_1a_2\cdots a_{n-1}}$. 
    \\¿Porqué? Considere por contradicción que $X'$ no es una solución óptima, entonces existe una 
    solución $Y'$ para este subproblema que codifica con menor tamaño que $X'$. Luego 
    $Y = Y' \cup \{n\}$ sería una solución mejor que X para el problema original, siendo esto
    una contradicción. Tomando en cuenta que debe haber una transición si $a_{n-1}$ no puede ser 
    codificado por el modo $C_j$.
\end{itemize}

\paragraph{Lemma 2.1:}
Sea $OPT(i, j)$ el tamaño de la codificación mínima óptima para el subproblema que considera a 
la subcadena $s'=\{a_1a_2 \cdots a_{i}\}$.

\[ OPT(i, j) \begin{cases} 
    B_j + C_j(a_i) & i = 1; a_i \: se\: \: puede\: codificar\: con\: C_j \\
    \infty & a_i \: no \: se\: \: puede\: codificar\: con\: C_j \\
    min\{OPT(i - 1, j) + C_j(a_i), & i > 1\\
    OPT(i - 1, (j+1) \bmod{4}) + T_{(j+1) \bmod{4}} + C_j(a_i), \\
    OPT(i - 1, (j+2) \bmod{4}) + T_{(j+2) \bmod{4}} +C_j(a_i), \\ 
    OPT(i - 1, (j+3) \bmod{4}) + T_{(j+3) \bmod{4}} + C_j(a_i)\}
 \end{cases}
\]

\paragraph{Prueba:}
Sea $X$ una solución óptima para el subproblema que considera a la subcadena $\{a_1a_2 \cdots a_{i}\}$.
Suponga que $a_i$ puede ser codificado por $C_j$. Sea $X' = X / \{a_i\}$. Note que $X'$ es una solución óptima para el
subproblema. (Vea análisis anterior). Por lo tanto:

\begin{itemize}
    \item Cuando $a_{i-1}$ puede ser codificado por el modo $C_j$. Entonces:
    \begin{center}
        $OPT(i,j) = OPT(i-1, j) + C_j(a_i)$
    \end{center}
    \item Cuando $a_{i-1}$ no puede ser codificado por el modo $C_j$. Se debe añadir una transición $T_k$ del modo
    $C_k$ donde $ 1 \leq k \leq 4$ que sí codifica $a_{i-1}$
    \begin{center}
        $ OPT(i,j) = min\{OPT(i - 1, (j+1) \bmod{4}) + T_{(j+1) \bmod{4}} + C_j(a_i)$, \\
        $OPT(i - 1, (j+2) \bmod{4}) + T_{(j+2) \bmod{4}} +C_j(a_i)$, \\ 
        $OPT(i - 1, (j+3) \bmod{4}) + T_{(j+3) \bmod{4}} + C_j(a_i)\}$
    \end{center}
\end{itemize}

Ahora suponga que $a_i$ no puede ser codificado por $C_j$. Entonces:
\begin{center}
    $OPT(i,j) = \infty$
\end{center}

El análisis anterior nos permite diseñar un algoritmo recursivo para el problema.

\newpage 

\subsection{Pregunta 3}

\noindent RECIBE: un índice final $i$ de una cadena $s$ y un índice $j$ de $C$.\\
DEVUELVE: Una cadena $r$ que codifica a $s$ de manera mínima óptima con $C_j(a_i)$.

\begin{algorithmic}[1]
\item[]{\textbf{OPT($i$, $j$)}}
\If{($a_i$ puede ser codificado por $C_j)$ y $i = 1$}
\State \textbf{return} $B_j + C_j(a_i)$
\EndIf
\If {($a_i$ no puede ser codificado por $C_j)$}
\State \textbf{return} $\infty$
\EndIf
\State $A_0 = C_j(a_i)$
\State $A_1 = T_{(j+1)\bmod{4}} + C_j(a_i)$
\State $A_2 = T_{(j+2)\bmod{4}} + C_j(a_i)$
\State $A_3 = T_{(j+3)\bmod{4}} + C_j(a_i)$
\State $r=min(OPT(i-1,j) + A_0, OPT(i-1,(j+1)\bmod{4}) + A_1, OPT(i-1,(j+2)\bmod{4}) + A_2, OPT(i-1,(j+3)\bmod{4}) + A_3)$
\State \textbf{return r} 
\end{algorithmic}

\vspace{0.5cm}

\noindent RECIBE: un cadena $s$ no vacía.\\
DEVUELVE: Una cadena que codifica a $s$ de manera mínima óptima.
\begin{algorithmic}[1]
\item[]{\textbf{MIN-COD($s$)}}
\State Sea $a_n$ el caracter final de $s$.  
\State \textbf{return} $min(OPT(a_n,0), OPT(a_n,1), OPT(a_n,2), OPT(a_n,3))$  
\end{algorithmic}

\subsubsection{Complejidad}

En \emph{OPT(i,j)} se realizan cuatro llamadas recursivas para intentar codificar con los modos $C_1,C_2,C_3,C_4$ 
a cada caracter de la cadena $s$. Por lo tanto, $T(n) = \Omega(4^n)$.\\

En \emph{MIN-COD(s)} se envía a \emph{OPT(i,j)} los cuatros modos diferentes con los que se puede codificar el caracter 
$a_n$. Sin embargo, aún cuando se llama cuatro veces a OPT, la recursividad total sigue siendo $\Omega(4^n)$.

\newpage

\subsection{Pregunta 4}

\noindent RECIBE: un índice final $i$ de una cadena $s$ y un índice $j$ de $C$.\\
DEVUELVE: Una cadena $r$ que codifica a $s$ de manera mínima óptima con $C_j(a_i)$.

\begin{algorithmic}[1]
\item[]{\textbf{OPT-MEMOIZADO($i$, $j$)}}
\If{i == 0 OR $A[i][j] \neq \infty$}
\State \textbf{return} $A[i][j]$
\EndIf
\If {$a_i$ puede ser codificado por $C_j$}
    \State $A_0 = $ OPT-MEMOIZADO$(i-1, j) + C_j(a_i)$
    \State $A_1 = $ OPT-MEMOIZADO$(i-1, (j+1)\bmod{4}) + T_{(j+1) \bmod{4}} + C_j(a_i)$
    \State $A_2 = $ OPT-MEMOIZADO$(i-1, (j+2)\bmod{4}) + T_{(j+2) \bmod{4}} + C_j(a_i)$
    \State $A_3 = $ OPT-MEMOIZADO$(i-1, (j+3)\bmod{4}) + T_{(j+3) \bmod{4}} + C_j(a_i)$
    \State $A[i][j] =min(A_0, A_1, A_2, A_3)$
\EndIf
\State \textbf{return} $A[i][j]$
\end{algorithmic}

\vspace{0.5cm}

\noindent RECIBE: un cadena $s$ no vacía.\\
DEVUELVE: Una cadena que codifica a $s$ de manera mínima óptima.
\begin{algorithmic}[1]
\item[]{\textbf{MIN-COD-MEM($s$)}}
\State $A[0][0] = 00$ 
\State $A[0][1] = 01$ 
\State $A[0][2] = 10$ 
\State $A[0][3] = 11$
\For{i = 1 to n}
    \State $A[i][0\cdots3] = \infty$ 
\EndFor 
\State Sea $a_n$ el caracter final de $s$.  
\State \textbf{return} $min(OPT(a_n,0), OPT(a_n,1), OPT(a_n,2), OPT(a_n,3))$  
\end{algorithmic}


\subsubsection{Complejidad}

En \emph{OPT(i,j)} se realizan cuatro llamadas recursivas para intentar codificar con los modos $C_1,C_2,C_3,C_4$ 
a cada caracter de la cadena $s$. Sin embargo, una vez que se haya calculado la solución $A[i][j]$ esta se guarda en 
una matriz $A$. Asumiendo que la concatenación de bits de las líneas 4-7 tiene costo $O(1)$ entonces la 
recurrencia memoizada es $O(n)$.

En \emph{MIN-COD(s)} se envía a \emph{OPT(i,j)} los cuatros modos diferentes con los que se puede codificar el caracter 
$a_n$. En la matriz $A$ se asignan los valores previos iniciales $B$, así como valores $\infty$ para el resto de la matriz.

\newpage

\subsection{Pregunta 5}
\noindent RECIBE: una cadena $s$.\\
DEVUELVE: Una cadena $r$ que codifica a $s$ de manera mínima óptima.

\begin{algorithmic}[1]
\item[]{MIN-COD-DIN($s$)}
\State Sea $A$ una matriz que guarda la solución.
\State $A[0][0] = 00$ 
\State $A[0][1] = 01$ 
\State $A[0][2] = 10$ 
\State $A[0][3] = 11$
\For{i = 1 to n}
    \State $A[i][1] = \infty$
    \State $A[i][2] = \infty$
    \State $A[i][3] = \infty$
    \State $A[i][4] = \infty$
    \For{j = 0 to 3}
        \If{Si $a_i$ puede ser codificado con $C_j$}
            \State $M_1 = A[i-1][j] + C_j(a_i)$
            \State $M_2 = A[i-1][(j+1)\bmod{4}] + T_{(j+1)\bmod{4}} + C_{j}(a_i)$
            \State $M_3 = A[i-1][(j+2)\bmod{4}] + T_{(j+2)\bmod{4}} + C_{j}(a_i)$
            \State $M_4 = A[i-1][(j+3)\bmod{4}] + T_{(j+3)\bmod{4}} + C_{j}(a_i)$
            \State $A[i][j] = min(M_1, M_2, M_3, M_4)$
            \State
        \EndIf
    \EndFor
\EndFor
\State $r = min(A[n][1], A[n][2], A[n][3], A[n][4])$
\State \textbf{return r}
\end{algorithmic}


\subsubsection{Complejidad}

En \emph{MIN-COD-DIN(s)} asumiendo que la concantenación de cadenas de bits es $O(1)$ es claro que el algoritmo es $\Omega(n)$. 
Los cuatros modos diferentes de la recurrencia aumentan el costo lineal en cuatro. La respuesta óptima del algoritmo está
en el valor mínimo de $A[n][1], A[n][2], A[n][3]$ y $A[n][4]$.

